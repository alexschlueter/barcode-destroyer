Kurze Einleitung

\subsection{Gradient+Blur}
Bla bla
\subsection{LSD}
This localization method is based on the LineSegmentDetector algorithm
\cite{GromponevonGioi2012}, which is implemented in OpenCV 3 in the
cv::LineSegmentDetector class \cite{Bradski2017}. The algorithm detects line
segments in an image, such as the segments formed by the bars in a barcode.
On a high level, it works by first calculating the image gradient and assigning
to each pixel a unit vector perpendicular to the gradient, resulting in a
\emph{level line field}. Pixels with a gradient magnitude under a certain
tolerance are discarded. Connected pixels with similar level line angles are clustered
together and form a \emph{line support region}. To each line support region, a
rectangle is fit which covers the whole region. To limit the number of false
detections, the rectangle is only accepted as a line segment if the number of
covered pixels with aligned level lines is high enough relative to the total number of
covered pixels. The exact threshold is based on a statistical model (\emph{a
  contrario} method), so that the rectangle is only accepted if,
in a purely random level line field, the found ratio of aligned level lines is unlikely.
Before a rectangle is rejected, some variations of the rectangle's parameters
are tried.

The barcode localization method is taken from \citeauthor{Creusot2016} with minor modifications. First we run
the LineSegmentDetector provided by OpenCV on the image. We use default
parameters, except for the first parameter LSD\_REFINE\_NONE, which disables
refinement of line segments. We found that line refinement tends to break up
barcode bars into multiple segments, e.g. if the bars in the image are not quite
straight due to kinks in the material on which the barcode is printed, or if
there are glare spots on the barcode. This is undesirable, since in the
following steps we rely on the length of line segments in the barcode to be similar.

Next we want to find a line segment which belongs to a barcode bar approximately
in the middle of the barcode. To each detected line segment, a score is assigned
based on how many other line segments might belong to bars of the same barcode as the first
one. Two line segments are regarded as possibly belonging to the same barcode, if
\begin{enumerate}
\item their centers are not too far apart,
\item they have similar length,
\item they have similar angles and
\item their projected intersection covers most of the smaller segment.
\end{enumerate}

Calculation of the first three distance measures is straightforward: Let $c_i$
be the center positions, $l_i$ the segment lengths and $\alpha_i$ the segment angles, where the index $i=1,2$
denotes the first or the second line segment, respectively. The used criteria are
\begin{equation*}
d_{\text{center}}\coloneqq\frac{\norm{c_1-c_2}}{l_1}<1,\qquad \frac{\abs{l_1-l_2}}{l_1}<0.3,\qquad \abs{\alpha_1-\alpha_2}<0.1\,.
\end{equation*}
As BIIIIIIILD shows, two lines fulfilling the first three criteria might
still not be arranged like barcode bars, if they are displaced along the line
segment direction. To catch these cases, we project the second line onto the
first and check that the intersection is large enough relative to the length of
the smaller line, see BIIIIIIILD2.

For each line segment which fulfills the criteria, \citeauthor{Creusot2016}
increment the score of the first line by one. Instead, we increment the score of the
first line segment by $1-d_{\text{center}}$. This effectively weights the score by the distance
between the line segments, which favors line segments which lie in
the middle of the barcode.

Finally, we select the line segment with the highest score. This line segment
should correspond to a barcode bar in the middle of the barcode.

%%% Local Variables:
%%% mode: latex
%%% TeX-master: "00Ausarbeitung.tex"
%%% End:
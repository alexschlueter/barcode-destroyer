\documentclass[12pt,a4paper]{article}

\usepackage[utf8]{inputenc}
\usepackage[english]{babel}
\author{Aleksej Matis, Alexander Schlüter, Kjeld Schmidt}
\title{Titel: Barcode Detection and Recognition in Non-constrained Systems}
\date{24. März 2017}
\makeatletter
\let\inserttitle\@title
\makeatother
\usepackage{graphicx} 
%\usepackage[T1]{fontenc}
%\usepackage{txfonts} %Schriftart Times New Roman
\usepackage{helvet} %Schrift Arial
\renewcommand{\familydefault}{\sfdefault}
\fontfamily{phv}\selectfont
\usepackage[left=2.5cm,right=2.5cm,top=2.5cm,bottom=2.5cm,includeheadfoot]{geometry}
\usepackage[onehalfspacing]{setspace}
\usepackage{mathtools,amssymb,amsthm} % Verbesserung von amsmath (die amsmath selbst lädt)
\usepackage{fancyhdr} 
\pagestyle{fancy} 
\rhead{\thepage} \chead{} \lhead{\inserttitle}
\rfoot{} \cfoot{} \lfoot{}  
\renewcommand{\headrulewidth}{0.5pt}
\usepackage[%
	backend=biber,
	sortlocale=auto,
	natbib,
	hyperref,
	style=numeric, % eine unvollständige Auswahl von Styles: ieee, numeric, apa
  maxcitenames=1
	]%
{biblatex}
% \renewcommand{\bibsection}{\section{Literatur}}
\addbibresource{literature.bib}
\usepackage{url}
\usepackage{hyperref}
\hypersetup{
    colorlinks,
    citecolor=black,
    filecolor=black,
    linkcolor=black,
    urlcolor=black
}
\usepackage{cleveref}

%-- charakteristische-Funktion-/Indikatorfunktion-Eins '\ind'
\usepackage{silence}
\WarningFilter{latexfont}{Size substitutions with differences}
\WarningFilter{latexfont}{Font shape `U/bbold/m/n' in size}
\DeclareSymbolFont{bbold}{U}{bbold}{m}{n}
\DeclareSymbolFontAlphabet{\mathbbold}{bbold}
\newcommand{\ind}{\mathbbold{1}}

%-- Ein sehr hübscher Mengen-Befehl
\newcommand\SetSymbol[1][]{\nonscript\:#1\vert\allowbreak\nonscript\:\mathopen{}}
\providecommand\given{} % to make it exist
\DeclarePairedDelimiterX\set[1]\{\}{\renewcommand\given{\SetSymbol[\delimsize]}#1}

%-- Klammern, Skalarprodukt und Norm
\DeclarePairedDelimiter{\enbrace}{(}{)}
\DeclarePairedDelimiter{\abs}{|}{|}
\DeclarePairedDelimiterX\skal[2]{\langle}{\rangle}{#1\,\delimsize\vert\,#2}
\DeclarePairedDelimiter{\norm}{\lVert}{\rVert}

\begin{document}


\thispagestyle{empty}
\begin{center}
\vspace*{2cm}
%i don't know if this numbers are correct, but it works and looks fine
\includegraphics[width=12cm,natwidth=12cm,natheight=3cm]{img/wwu-logo-neu.pdf}\\
\vspace*{2cm}
\Large
\textbf{\inserttitle}\\
\normalsize
\vspace*{2cm}
\textbf{Ausarbeitung}\\
\vspace*{1cm}
Veranstaltung:\\
\textbf{Begleitendes Praktikum zu Computer Vision WS 2016/17}
\end{center}
\vfill

\begin{center}
\begin{tabular}{ll}
Themensteller:&\textbf{Prof. Dr. Xiaoyi Jiang}\\
&Dimitri Berh\\
&Andreas Nienkötter\\
Betreuer:&Aaron Scherzinger\\
Verfasser:&\textbf{Aleksej Matis}\\
&\textbf{Alexander Schlüter}\\
&\textbf{Kjeld Schmidt}
\end{tabular} 
\end{center}
\setcounter{page}{0}
\maketitle
\thispagestyle{empty} 
\tableofcontents
\newpage
\section{Einleitung} 
The goal of this project is to read barcodes in given images. The barcodes can
be arbitrarily translated, rotated and scaled in the image. The perspective
might be skewed and parts of the image might be out of focus or suffer from bad
lighting and glare spots. These images are typically obtained by mobile phone
cameras, without requiring the user to center the barcode in the image, or to
wait until the autofocus is fully adjusted.

There are no time or performance constrains placed on our detection.

Let us first describe the structure of an EAN-13 barcode \cite{GS12017}:
The smallest width of a bar is called the base width. A typical digit is encoded
in four bars, alternating black and white, where the number of base widths
of each bar is variable, but the total number of base widths of all four bars is seven.
The exact number of base widths of each bar encodes a specific digit as well as a digit
type, one of A, B or C.

A barcode is now composed as follows:
\begin{enumerate}
\item A left quiet zone,
\item a guard pattern, consisting of three bars, each one base width,
\item six digits of type A or B,
\item a center guard pattern, consisting of five bars, each one base width,
\item six digits of type C,
\item a guard pattern, same as 2.,
\item a right quiet zone.
\end{enumerate}

The total number of base widths in a barcode is thus $3 + 6*7 + 5 + 6*7 + 3=95$.
The pattern types (A or B) of the first six read digits encode the actual first
digit of the barcode. The last digit is a check digit and has to match the digit
obtained through a calculation based on the other digits.

We implemented and compared multiple methods. The methods can be divided into
three general steps:
\begin{enumerate}
\item \textbf{Localization}: finding the general location and orientation of the barcode
  in the image
\item \textbf{Boundary Detection}: determining the barcode's left and right boundaries
\item \textbf{Reading}
\end{enumerate}

%%% Local Variables:
%%% mode: latex
%%% TeX-master: "00Ausarbeitung.tex"
%%% End:
\newpage

\section{Lokalisierung} 
In order to read a barcode, we first have to localize it in a given image. We implemented two approaches. Gradient+Blur is a simple algorithm, with which we had some initial success. We later discarded it in favor of the much more successful Line Segment Detector approach. Both are detailed in this chapter.

\subsection{Gradient+Blur}
Our initial localization attempt showed limited success, but is conceptually very simple. The Gradient+Blur approach seeks to exploit the strong boundaries between the (white) background area and black bars in a standard barcode. In a few words; Detect edges on the image, then blur the result to remove noise and apply a threshold to the result. Open and close the resulting shapes, then select the biggest one. We will now go trough the steps in more detail.

\begin{enumerate}
	\item Edge Detection
	
	We begin by detecting edges with the Sobel operator. We calculate the absolute gradient, discarding direction information. This leads to slightly worse results for barcodes rotated by about 45 degrees. The resulting image will usually contain large amounts of noise. This is fixed in the next step. 

	\item Blur+Threshold
	
	Next, we blur the previous result to smooth out the noise and apply a threshold afterwards. The blur is a normalized box filter of size 9. Note that 'noise' here refers not only to image imperfections, but also noise in the absolute gradient which might be a result of complex textures or other sources. The threshold is not set as a constant brightness of 127. Instead, we calculate the mean brightness of the image, add that value to 255 and divide by 2. This simple extra step gives slightly better results when dealing with very noisy images.
	
	\item Close Forms
	
	The next two steps are very similar operations. First we \emph{close} the remaining white areas. This means we first dilate the image, then erode it by the same amount. Dilation creates a new image in which a pixel is set to white if there is at least one white pixel within a given radius white in the original image. More intuitively, white pixels grow outwards. Erosion is the opposite operation; A pixel is set to black, unless all pixels in the radius are white.  This \emph{closes} empty spaces between neighboring and inside of structures, while structures with no neighbors will remain unchanged. At this point, the barcode will ideally have merged into a solid shape, 
	
	\item Open Forms
	
	Opening the forms is similar to closing, but in reverse order; First eroding, then dilating by the same amount. This will eliminate small structures (which is not actually impotant at this point), while straightening edges on larger structures. 
	
	\item Selection
	
	Finally, we select the largest structure to be the most likely location of the barcode in the original image. In most cases, our result will be a jagged shape, only roughly tracing a rectangle. To simplify matters going forward, we now find the minimal area rectangle around the selected structure. This will get passed into the boundary detection step.
\end{enumerate}

\subsection{LSD}\label{sec:LSD}
This localization method is based on the LineSegmentDetector algorithm
\cite{GromponevonGioi2012}, which is implemented in OpenCV 3 in the
cv::LineSegmentDetector class \cite{Bradski2017}. The algorithm detects line
segments in an image, such as the segments formed by the bars in a barcode.
On a high level, it works by first calculating the image gradient and assigning
to each pixel a unit vector perpendicular to the gradient, resulting in a
\emph{level line field}. Pixels with a gradient magnitude under a certain
tolerance are discarded. Connected pixels with similar level line angles are clustered
together and form a \emph{line support region}. To each line support region, a
rectangle is fit which covers the whole region. To limit the number of false
detections, the rectangle is only accepted as a line segment if the number of
covered pixels with aligned level lines is high enough relative to the total number of
covered pixels. The exact threshold is based on a statistical model (\emph{a
  contrario} method), so that the rectangle is only accepted if,
in a purely random level line field, the found ratio of aligned level lines is unlikely.
Before a rectangle is rejected, some variations of the rectangle's parameters
are tried.

The barcode localization method is taken from \citeauthor{Creusot2016} with minor modifications. First we run
the LineSegmentDetector provided by OpenCV on the image. We use default
parameters, except for the first parameter LSD\_REFINE\_NONE, which disables
refinement of line segments. We found that line refinement tends to break up
barcode bars into multiple segments, e.g. if the bars in the image are not quite
straight due to kinks in the material on which the barcode is printed, or if
there are glare spots on the barcode. This is undesirable, since in the
following steps we rely on the length of line segments in the barcode to be similar.

Next we want to find a line segment which belongs to a barcode bar approximately
in the middle of the barcode. To each detected line segment, a score is assigned
based on how many other line segments might belong to bars of the same barcode as the first
one. Two line segments are regarded as possibly belonging to the same barcode, if
\begin{enumerate}
\item their centers are not too far apart,
\item they have similar length,
\item they have similar angles and
\item their projected intersection covers most of the smaller segment.
\end{enumerate}

Calculation of the first three distance measures is straightforward: Let $c_i$
be the center positions, $l_i$ the segment lengths and $\alpha_i$ the segment angles, where the index $i=1,2$
denotes the first or the second line segment, respectively. The used criteria are
\begin{equation*}
d_{\text{center}}\coloneqq\frac{\norm{c_1-c_2}}{l_1}<1,\qquad \frac{\abs{l_1-l_2}}{l_1}<0.3,\qquad \abs{\alpha_1-\alpha_2}<0.1\,.
\end{equation*}
As BIIIIIIILD shows, two lines fulfilling the first three criteria might
still not be arranged like barcode bars, if they are displaced along the line
segment direction. To catch these cases, we project the second line onto the
first and check that the intersection is large enough relative to the length of
the smaller line, see BIIIIIIILD2.

For each line segment which fulfills the criteria, \citeauthor{Creusot2016}
increment the score of the first line by one. Instead, we increment the score of the
first line segment by $1-d_{\text{center}}$. This effectively weights the score by the distance
between the line segments, which favors line segments which lie in
the middle of the barcode.

Finally, we select the line segment with the highest score. This line segment
should correspond to a barcode bar in the middle of the barcode.

%%% Local Variables:
%%% mode: latex
%%% TeX-master: "00Ausarbeitung.tex"
%%% End:
\newpage

\section{Boundary Detection} 
Einleitung
\subsection[Wachenfeld]{Wachenfeld \cite{wachenfeld2008robust}}
\begin{figure}[t]
\center
\includegraphics[width=0.6\textwidth,natwidth=900,natheight=463]{img/wachenfeld.png}
\caption{Wachenfeld}
\label{wachenfeld}
\end{figure}
Abbildun \ref{wachenfeld}

\subsection{LSD Bound}
bla bla
\subsection{Variation}
machen wir das?
\newpage

\section{Lesen} 
% Template Matching is a general purpose method usually used for object
% recognition. A template is created for the sought object and an image is scanned
% with the template, calculating a matching cost for each possible position.

% Naively applying template matching to the barcode reading problem would not
% yield an interesting algorithm, because after choosing a scanline, reading is
% reduced to finding a barcode such that the 

Our reading method is taken from \cite{Gallo2011} and is based on matching
deformable templates to a 1D scanline obtained from the localization and
boundary detection steps.

Let us first describe the structure of an EAN-13 barcode \cite{GS12017}:
The smallest width of a bar is called the base width. A typical digit is encoded
in four bars, alternating black and white, where the number of base widths
of each bar is variable, but the total number of base widths of all four bars is seven.
The exact number of base widths of each bar encodes a specific digit as well as a digit
type, one of A, B or C.

A barcode is now composed as follows:
\begin{enumerate}
\item A left quiet zone,
\item a guard pattern, consisting of three bars, each one base width,
\item six digits of type A or B,
\item a center guard pattern, consisting of five bars, each one base width,
\item six digits of type C,
\item a guard pattern, same as 2.,
\item a right quiet zone.
\end{enumerate}

The pattern types (A or B) of the first six read digits encode the actual first
digit of the barcode. The last digit is a check digit and has to match the digit
obtained through a calculation based on the other digits.
%%% Local Variables:
%%% mode: latex
%%% TeX-master: "00Ausarbeitung.tex"
%%% End:
 
\newpage

\section{Vergleich} 
Einleitung...
\subsection{Datasets}
The implemented algorithms were tested on three datasets.
\begin{itemize}
\item Generated Barcodes:
\begin{itemize}
\item 110 images, 300$\times$150 px, barcode only.
\item 110 images, 1000$\times$1000 px, barcodes (300$\times$150 px) randomly rotated and translated.
\end{itemize}
\item WWU Muenster Barcode Database \cite{MuensterBarcodeDB} \citep{wachenfeld2008robust}:\\
1055 images, 800$\times$600 px.
\item ArteLab Dataset - Robust Angle Invariant 1D Barcode Detection \cite{ArteLabDB} \cite{zamberletti2010neural} \citep{zamberletti2013robust}:\\
2 sets of 215 images, 800$\times$600 px.
\end{itemize}
\subsection{Laufzeit}
Geprüft auf dem Wachenfeld Dataset mit 1055 Bildern \cite{wachenfeld2008robust}\\
Laufzeit gemessen auf Debian 8 mit 4Gb Ram, 2.2 Ghz CPU (Intel N2940), 4 Kerne und 4 Threads

\begin{figure}[t]
\center
\bgroup
\def\arraystretch{1.5}
\begin{tabular}{|l|r|r|r|}
\hline
&\textbf{Errors}&\textbf{Hit rate}&\textbf{Time in sec.}\\
\hline
\textbf{Gradient + Blur}& 831& 22\%& 266\\
\hline
\textbf{LSD + Wachenfeld}& 452& 56\%& 319\\
\hline
\textbf{LSD + LSDBounds}& 41& 97\%& 1760\\
\hline
\textbf{LSD + Variation}& 279& 74\%& 1595\\
\hline
\end{tabular}
\egroup
\caption{Comparison of different algorithms on the Wachenfeld database \citep{MuensterBarcodeDB}}
\label{laufzeit}
\end{figure}
Siehe \cref{laufzeit}
 
\newpage

\section{Verbesserung} 
In this section, we will consider some possible improvements of our algorithms.

\subsection{Gradient+Blur Detection}

\subsubsection*{Vary by rotation}
Since the Gradient+Blur-detection builds from the absolute gradients found by the Sobel operator, results might be poor when a barcode is rotated close to 45 degrees. To remedy this, we might run the detection on rotated versions of the image. Rotating once by 45 degrees would yield the greatest improvement, however, multiple rotations could be checked at diminishing returns.

\subsubsection*{Vary kernel sizes}
In our current implementation, all operations (blurring, closing, opening) are applied with a constant kernel size. This is somewhat problematic, since the effectiveness of these parameters is dependent on the size of the barcode. If a barcode is not detected successfully, we might try different kernel sizes.\newline
The initial kernel size might also be considered further: The current values have been empirically determined to be most useful for the image sizes we worked with. Scaling the kernel size appropriately for each image might lead to better results, since it seems likely that the relative sizes barcodes in images is independent of image resolution.


\subsection{LSD}
\subsubsection*{Unite interrupted line segments}

A remaining common failure of the LSD algorithm is it's poor handling of reflections on the barcode. These reflections visually interrupt the barcode lines, so two short line segments are returned instead of one. This is problematic, as length of line segments is an important factor in deciding whether two line segments might be part of the same barcode. This ultimately leads to a split in the barcode, where the detected area covers just the larger side of the code.

This problem could be fixed in two related ways: First, we could try a \emph{line segment unification}. This process would seek to unite line segments which lie on the same line in space. This easy calculation would unite line segments across gaps, thus restoring the proper length and avoiding rejection. To avoid uniting many stray lines across the image, simply limit unification to line segments where the gap size is less than some small multiple of the segment size itself.

This first approach has the weakness of only working when the reflection splits the line segment into two parts. Even better would be an approach that could handle a missing endpoint. This should be handled by this second approach, which is also more complicated: Instead of uniting line segments, we would now consider the endpoints of line segments in a second pass over all segments. During the first pass, whenever a pair of lines is considered to likely be part of the same barcode, save the location of their endpoints in two different lists, based on distance from the origin. After this first pass, those lists can easily construct the upper and lower bound of the barcode. With this new information, discard the previous scores and redo them. However, this time, instead of checking for similar length, check whether two line segments have an endpoint near one of those two lines before assigning a score to them. This second pass will now also handle missing endpoints.

\subsection{Wachenfeld}
\subsubsection*{Proper preprocessing, multiple scanlines}

Since we have found quick success with the LSD-Boundary detection, our current implementation of the Wachenfeld algorithm does not prepare the image in the way recommended in the original paper, that is, there is no adaptive thresholding along the scanline, which is a major factor of success for the original implementation. Thus, proper initialization would likely greatly improve our results. Also, we could detect along multiple scanlines and vote for most likely boundaries.

\subsection{Template Matching}
\subsubsection*{Precompute patterns for blur of various strengths}

While the algorithm is already very capable in handling blurred barcodes, this might be improved even further by precomputing patterns specifically not for an ideal, clean barcode, but for one that is already blurred.

\newpage

\section{Fazit} 
In this work we have seen how different algorithms compare against each other in detecting barcodes.

After some initial comparisons of Gradient+Blur and LSD we decided to go with the latter because of the superior localization quality. And since we were satisfied with the high quality of the readings performed by the Template Matching we decided to focus our efforts on improving the overall hit rate by implementing additional boundary detection algorithms.

While comparing the Wachenfeld boundary detection against the Variation boundary detection we have shown that by sacrificing speed and increasing complexity we can increase the detection rate. Since we wanted to maximize the overall hit rate with regards to the competition within the scope of the practical course we improved on the Variation boundary detector by sacrificing more time and resources in favor of the LSDBound algorithm.

Obviously our methodology is not applicable for real time usage in the wild. A more fitting approach would be to implement a fast and simple algorithm like gradient blur enhanced with additional scan line based refinement like the Wachenfeld algorithm with optimized preprocessing and reading step. In a real time application the algorithm could depend on the user to provide sharp and disturbance free images with centered barcodes without rotation. 
\\
\\
Finally our work can be summarized with: Higher accuracy detection and recognition of barcodes in a non-constrained system takes more time and resources.
\newpage


\section{Meta: Wie zitiere ich?} 
\begin{enumerate}
\item Titel des Papers bei \url{https://scholar.google.de/} Google Scholar suchen.
\item Bei dem Eintrag zu dem Paper unten auf zitieren klicken, dann auf BibTex.
\item Den BibTex string kopieren in die \textsc{literatur.bib}
\item Zitat hinzufügen durch \textsc{\textbackslash cite\{\textit{name}\}}
\item Übersicht BibTex: \url{https://de.wikibooks.org/wiki/LaTeX-Kompendium:_Zitieren_mit_BibTeX}
\end{enumerate}
Beispiel Templatematching \cite{chen2014scanning}
%\input{03titel}

Einleitung
Lokalisierung
Rand
Lesen
Testdaten
Vergleich der Verfahren
Verbesserung



%\newpage
%\section{Einleitung} 
%\input{04titel}

\newpage
\appendix
\printbibliography
\end{document}

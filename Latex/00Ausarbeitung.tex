\documentclass[12pt,a4paper]{article}

\usepackage[utf8]{inputenc}
\usepackage[english, german]{babel}
\author{Aleksej Matis, Alexander Schlüter, Kjeld Schmidt}
\title{Titel: Iwas mit Barcodes}
\date{24. März 2017}
\makeatletter
\let\inserttitle\@title
\makeatother
\usepackage{url}
\usepackage{graphicx} 
%\usepackage[T1]{fontenc}
%\usepackage{txfonts} %Schriftart Times New Roman
\usepackage{helvet} %Schrift Arial
\renewcommand{\familydefault}{\sfdefault}
\fontfamily{phv}\selectfont
\usepackage[left=2.5cm,right=2.5cm,top=2.5cm,bottom=2.5cm,includeheadfoot]{geometry}
\usepackage[onehalfspacing]{setspace}
\usepackage{amsmath}
\usepackage{fancyhdr} 
\pagestyle{fancy} 
\rhead{\thepage} \chead{} \lhead{\inserttitle}
\rfoot{} \cfoot{} \lfoot{}  
\renewcommand{\headrulewidth}{0.5pt}
\usepackage[authoryear]{natbib}
\renewcommand{\bibsection}{\section{Literatur}}
\usepackage{hyperref}
\hypersetup{
    colorlinks,
    citecolor=black,
    filecolor=black,
    linkcolor=black,
    urlcolor=black
}

\begin{document}


\thispagestyle{empty}
\begin{center}
\vspace*{2cm}
%i don't know if this numbers are correct, but it works and looks fine
\includegraphics[width=12cm,natwidth=12cm,natheight=3cm]{img/wwu-logo-neu.pdf}\\
\vspace*{2cm}
\Large
\textbf{\inserttitle}\\
\normalsize
\vspace*{2cm}
\textbf{Ausarbeitung}\\
\vspace*{1cm}
Veranstaltung:\\
\textbf{Begleitendes Praktikum zu Computer Vision WS 2016/17}
\end{center}
\vfill

\begin{center}
\begin{tabular}{ll}
Themensteller:&\textbf{Prof. Dr. Xiaoyi Jiang}\\
&Dimitri Berh\\
&Andreas Nienkötter\\
Betreuer:&Aaron Scherzinger\\
Verfasser:&\textbf{Aleksej Matis}\\
&\textbf{Alexander Schlüter}\\
&\textbf{Kjeld Schmidt}
\end{tabular} 
\end{center}
\setcounter{page}{0}
\maketitle
\thispagestyle{empty} 
\tableofcontents
\newpage
\section{Einleitung} 
The goal of this project is to read barcodes in given images. The barcodes can
be arbitrarily translated, rotated and scaled in the image. The perspective
might be skewed and parts of the image might be out of focus or suffer from bad
lighting and glare spots. These images are typically obtained by mobile phone
cameras, without requiring the user to center the barcode in the image, or to
wait until the autofocus is fully adjusted.

There are no time or performance constrains placed on our detection.

Let us first describe the structure of an EAN-13 barcode \cite{GS12017}:
The smallest width of a bar is called the base width. A typical digit is encoded
in four bars, alternating black and white, where the number of base widths
of each bar is variable, but the total number of base widths of all four bars is seven.
The exact number of base widths of each bar encodes a specific digit as well as a digit
type, one of A, B or C.

A barcode is now composed as follows:
\begin{enumerate}
\item A left quiet zone,
\item a guard pattern, consisting of three bars, each one base width,
\item six digits of type A or B,
\item a center guard pattern, consisting of five bars, each one base width,
\item six digits of type C,
\item a guard pattern, same as 2.,
\item a right quiet zone.
\end{enumerate}

The total number of base widths in a barcode is thus $3 + 6*7 + 5 + 6*7 + 3=95$.
The pattern types (A or B) of the first six read digits encode the actual first
digit of the barcode. The last digit is a check digit and has to match the digit
obtained through a calculation based on the other digits.

We implemented and compared multiple methods. The methods can be divided into
three general steps:
\begin{enumerate}
\item \textbf{Localization}: finding the general location and orientation of the barcode
  in the image
\item \textbf{Boundary Detection}: determining the barcode's left and right boundaries
\item \textbf{Reading}
\end{enumerate}

%%% Local Variables:
%%% mode: latex
%%% TeX-master: "00Ausarbeitung.tex"
%%% End:

\newpage
\section{Meta: Wie zitiere ich?} 
\begin{enumerate}
\item Titel des Papers bei \url{https://scholar.google.de/} Google Scholar suchen.
\item Bei dem Eintrag zu dem Paper unten auf zitieren klicken, dann auf BibTex.
\item Den BibTex string kopieren in die \textsc{literatur.bib}
\item Zitat hinzufügen durch \textsc{\textbackslash cite\{\textit{name}\}}
\item Übersicht BibTex: \url{https://de.wikibooks.org/wiki/LaTeX-Kompendium:_Zitieren_mit_BibTeX}
\end{enumerate}
Beispiel Templatematching \cite{chen2014scanning}
%\input{03titel}

%\newpage
%\section{Einleitung} 
%\input{04titel}

\newpage
\appendix
\begin{small}
\bibliography{literature}
\bibliographystyle{plain}
\end{small}

\end{document}

% Template Matching is a general purpose method usually used for object
% recognition. A template is created for the sought object and an image is scanned
% with the template, calculating a matching cost for each possible position.

% Naively applying template matching to the barcode reading problem would not
% yield an interesting algorithm, because after choosing a scanline, reading is
% reduced to finding a barcode such that the 

Our reading method is taken from \cite{Gallo2011} and is based on matching
deformable templates to a 1D scanline obtained from the localization and
boundary detection steps.

Let us first describe the structure of an EAN-13 barcode \cite{GS12017}:
The smallest width of a bar is called the base width. A typical digit is encoded
in four bars, alternating black and white, where the number of base widths
of each bar is variable, but the total number of base widths of all four bars is seven.
The exact number of base widths of each bar encodes a specific digit as well as a digit
type, one of A, B or C.

A barcode is now composed as follows:
\begin{enumerate}
\item A left quiet zone,
\item a guard pattern, consisting of three bars, each one base width,
\item six digits of type A or B,
\item a center guard pattern, consisting of five bars, each one base width,
\item six digits of type C,
\item a guard pattern, same as 2.,
\item a right quiet zone.
\end{enumerate}

The pattern types (A or B) of the first six read digits encode the actual first
digit of the barcode. The last digit is a check digit and has to match the digit
obtained through a calculation based on the other digits.
%%% Local Variables:
%%% mode: latex
%%% TeX-master: "00Ausarbeitung.tex"
%%% End:
